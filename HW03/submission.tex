\documentclass[]{article}

\author{Rushi Shah}
\date{\today}
\title{Assignment 2}

\setlength{\columnsep}{.75in}

\usepackage{amsmath}
\usepackage{amsthm}
\usepackage{amssymb}
\usepackage{mathtools}

\usepackage[margin=1in]{geometry}

\usepackage{titling}
\setlength{\droptitle}{-85pt}

% Prefix section headers with "Problem"
\renewcommand*{\thesection}{Problem~\arabic{section}}


% Note: `\to` is a synonym for `\rightarrow`
\newcommand{\integers}{\mathbb{Z}}
\newcommand{\naturals}{\mathbb{N}}
\newcommand{\reals}{\mathbb{R}}
\newcommand{\complexes}{\mathbb{C}}
\newcommand{\rationals}{\mathbb{Q}}
\newcommand{\inv}{^{-1}}
\DeclarePairedDelimiter\floor{\lfloor}{\rfloor}


\begin{document}
	\maketitle



	\section{}
		\textit{Prove that the set $S \equiv \{5, 10, 15, 20, \cdots\}$ is countable by constructing a one-to-one function from $S$ onto $\naturals$}

			First note that $S = \{5, 10, 15, 20, \cdots\} = \{5k\ |\ k \in \naturals\}$. Thus, we can construct a one-to-one function $f : S \to \naturals$ by defining $f(5k) \mapsto k$. 


	\section{}
		\subsection*{Part a}
			\emph{The union of two finite sets is finite.}

			Consider two finite sets: $X$ with cardinality $n$ and $Y$ with cardinality $m$ such that WLOG $X = \{1,\ldots,n\}$ and $Y = \{1,\ldots,m\}$. Then there exists a function with $Dom(\{1,\ldots,n\} \cup \{1,\ldots,m\})$ that is bijective with $\{1, \ldots, n + m\}$
			\[f(z) = \begin{cases}
				z, z \in \{1,\ldots,n\} \\
				n + z, z \in \{1,\ldots,m\}
			\end{cases}\]
			Thus, $X \cup Y$ is finite with cardinality $n + m$. 
		\subsection*{Part b}
			\emph{The union of a finite set and a countable set is countable.}

			Consider a finite set that has cardinality n which (WLOG) can be stated as $X = \{1, \ldots, n\}$ and a countable set which (WLOG) can be stated as $Y = \naturals$. Then we can define a bijection $f : X \cup Y \to \naturals$

			\[f(z) = \begin{cases}
				z, z \in X \\
				n + z, z \in Y
			\end{cases}\]

			Thus $X \cup Y$ is countable. 
		\subsection*{Part c}
			\emph{The union of two countable sets is countable.}

			Consider two countable sets $X = \{x_1, x_2, \ldots\}$ and $Y = \{y_1, y_2, \ldots\}$. Then we can define a bijection $f : \naturals \to X \cup Y$

			\[f(z) = \begin{cases}
				x_z, z \text{ is even} \\
				y_z, z \text{ is odd}
			\end{cases}\]

			Thus $X \cup Y$ is countable. 

	\section{}
		\begin{em}
			Rational numbers are defined as real numbers that can be written in the form $\frac{m}{n}, n \neq 0$ with m and n integers without common factors. The set of rational numbers, $\rationals$, can be split into three parts, the positive ones $\rationals_+$, the negative ones $\rationals_-$, and the set that contains only zero $\{0\}$, $\rationals = \rationals_+ \cup \{0\} \cup \rationals_-$. Prove that $\rationals$ is countable.

			Hint: We can first show that $\rationals_+$ is countable by constructing the function $f : \frac{m}{n} \mapsto (m, n), f : \rationals \mapsto U \subset \naturals \times \naturals$ that is one-to-one with $Dom(f) = \rationals_+$. $\naturals \times \naturals$ is countable, so U is countable, so $\rationals_+$ is countable. We then use the results in the previous problem 2. 
		\end{em}

		We can define $f_1 : \rationals_+ \to U_1 \subset \naturals \times \naturals$ by $f_1(\frac{m}{n}) \mapsto (m, n)$. Note that negative rationals can all be represented with positive denominators by multiplying any negative rational with a negative denominator by $\frac{-1}{-1} = 1$. Thus, we can define $f_2 : \rationals_- \to U_2 \subset \naturals \times \naturals$ by $f_2(\frac{m}{n}) \mapsto (|m|,n)$. With these two functions, we see that $\rationals_-, \rationals_+$ are both countable. Note that it is clear that $\{0\}$ is finite. Since the union of countable sets and the union of finite sets are countable by the previous problem, we can see that $\rationals = \rationals_+ \cup \{0\} \cup \rationals_-$ is countable.

		% is (-3)/2 the same thing as 3/(-2) ? 

	\section{}
		\emph{Let $\rho$ be a metric on X. Prove that the following are also metrics}
		\subsection*{Part a} 
			$\rho_1 \equiv 5 \rho$ 

			We must show that $\rho_1$ satisfies non-negativity, symmetry, and the triangle inequality. 

			\underline{Non-negativity:}
				Take arbitrary $x, y \in X$. We know that $\rho(x, y) = 0$ iff $x = y$. But $5 \rho(x, y) = 0$ iff $\rho(x, y) = 0$ so $\rho_1(x, y) = 0$ iff $x = y$. Similarly, $5\rho(x, y) \geq 0$ when $\rho(x, y) \geq 0$. Since $\rho(x, y) \geq 0\ \forall\ x, y \in X$, we know that $\rho_1$ satisfies non-negativity. 

			\underline{Symmetry:}
				Take arbitrary $x, y \in X$: 
				\[\rho_1(x, y) = 5 \rho(x, y) = 5 \rho(y, x) = \rho_1(x, y)\]

			\underline{Triangle inequality:} 
				Take arbitrary $x, y, z \in X$:

				\begin{align*}
					\rho(x, y) &\leq \rho(x, z) + \rho(z, y) \\
					5 \cdot \rho(x, y) &\leq 5 \cdot (\rho(x,z) + \rho(z, y)) \\
					\rho_1(x, y) &\leq 5\rho(x, z) + 5\rho(z, y) \\
					&\leq \rho_1(x, z) + \rho_1(z, y)
				\end{align*}

		\subsection*{Part b}
			$\rho_2 \equiv min\{1, \rho\}$

			\underline{Non-negativity:}

				Since $0 < 1$, $\rho_2(x, y) = 0$ iff $\rho(x, y) = 0$ iff $x = y$. Similarly, since $\rho$ is never less than zero, and 1 is positive, $\rho_2(x, y) \geq 0\ \forall\ x, y\in X$. 

			\underline{Symmetry:}

				\[
					\rho_2(x, y) = min(1, \rho(x, y)) = min(1, \rho(y, x)) = \rho_2(x, y)
				\]

			\underline{Triangle inequality:}

				% TODO

				We would like to show that $\rho_2(x, y) \leq \rho_2(x, z) + \rho_2(z, y)$

				\begin{enumerate}
					\item $min(1, \rho(x, y)) = 1, min(1, \rho(x, z)) = 1, min(1, \rho(z, y)) = 1, 1 \leq 1 + 1$. 

					\item $min(1, \rho(x, y)) = \rho(x, y), min(1, \rho(x, z)) = 1, min(1, \rho(z, y)) = 1, \rho(x, y) \leq \rho(x, z) + \rho(z, y)$ and $\rho(x, z) > 1, \rho(z, y) > 1$, so $\rho(x, y) \leq 1 + 1$. 

					\item $min(1, \rho(x, y)) = 1, min(1, \rho(x, z)) = \rho(x, z), min(1, \rho(z, y)) = 1$. $\rho(x, z) > 1$ gives us $1 \leq 1 + \rho(x, z)$.

					\item $min(1, \rho(x, y)) = 1, min(1, \rho(x, z)) = 1, min(1, \rho(z, y)) = \rho(z, y)$. WLOG from \#3.

					\item $min(1, \rho(x, y)) = \rho(x, y), min(1, \rho(x, z)) = \rho(x, z), min(1, \rho(z, y)) = 1$, $\rho(x, y) \leq \rho(x, z) + \rho(z, y)$ and $\rho(z, y) > 1$, so $\rho(x, y) \leq \rho(x, z) + 1$ 

					\item $min(1, \rho(x, y)) = \rho(x, y), min(1, \rho(x, z)) = 1, min(1, \rho(z, y)) = \rho(z, y)$ WLOG from \# 5

					\item $min(1, \rho(x, y)) = 1, min(1, \rho(x, z)) = \rho(x, z), min(1, \rho(z, y)) = \rho(z, y)$, $1 < 1 + 1$ and $\rho(x, z) \geq 1, \rho(z, y) \geq 1$, so $1 < \rho(x, z) + \rho(z, y) $

					\item $min(1, \rho(x, y)) = \rho(x, y), min(1, \rho(x, z)) = \rho(x, z), min(1, \rho(z, y)) = \rho(z, y)$, triangle inequality from $\rho$. 
				\end{enumerate}

	\section{}
		\textit{Let X = (0, 1]. Define $\rho(x, y) \equiv \left | \frac{1}{x} - \frac{1}{y} \right |$. Prove that $\rho$ is a metric on X.}

			\underline{Non-negativity:}

				If $x = y$ then $\rho(x, y) = \left | \frac{1}{x} - \frac{1}{y} \right | = \left | \frac{1}{x} - \frac{1}{x} \right | = \left | 0 \right | = 0$. 

				If $\rho(x, y) = 0$ then $x = y$: 

				\begin{align*}
					\rho(x, y) &= 0 \\
					\left | \frac{1}{x} - \frac{1}{y} \right | &= 0 \\
					\frac{1}{x} &= \frac{1}{y} \\
					x &= y
				\end{align*}

			\underline{Symmetry:}

				\[\rho(x, y) = \left | \frac{1}{x} - \frac{1}{y}\right | = \left | - \left (\frac{1}{y} - \frac{1}{x} \right ) \right | = \left | \frac{1}{y} - \frac{1}{x} \right | = \rho(y, x)\]

			\underline{Triangle inequality:}

				% Absolute value triangle inequality: |a + b| ≤ |a| + |b|
				For all $x, y, z \in X$, we must show that $\rho(x, y) \leq \rho(x, z) + \rho(z, y)$. 

				\begin{align*}
					\rho(x,y) &= \left | \frac{1}{x} - \frac{1}{y} \right | \\
					&= \left | \frac{1}{x} - \frac{1}{z} + \frac{1}{z} - \frac{1}{y} \right | \\
					&= \left | \left (\frac{1}{x} - \frac{1}{z}\right ) + \left (\frac{1}{z} - \frac{1}{y}\right ) \right | \\ 
					&\leq \left | \frac{1}{x} - \frac{1}{z} \right | + \left | \frac{1}{z} - \frac{1}{y} \right | = \rho(x, z) + \rho(z, y)
				\end{align*}


\end{document}
