\documentclass[]{article}

\author{Rushi Shah}
\date{\today}
\title{Assignment 9}

\setlength{\columnsep}{.75in}

\usepackage{amsmath}
\usepackage{amsthm}
\usepackage{amssymb}
\usepackage{mathtools}

\usepackage[margin=1in]{geometry}

\usepackage{titling}
\setlength{\droptitle}{-85pt}

% Prefix section headers with "Problem"
\renewcommand*{\thesection}{Problem~\arabic{section}}


% Note: `\to` is a synonym for `\rightarrow`
\newcommand{\integers}{\mathbb{Z}}
\newcommand{\naturals}{\mathbb{N}}
\newcommand{\reals}{\mathbb{R}}
\newcommand{\complexes}{\mathbb{C}}
\newcommand{\rationals}{\mathbb{Q}}
\newcommand{\inv}{^{-1}}
\DeclarePairedDelimiter\floor{\lfloor}{\rfloor}


\begin{document}
	\maketitle

	\section{}
		\begin{em}
			The function $f$ defined as follows is called the Dirichlet function:

			\[f(x) = \begin{cases}
				1\ x \in \rationals \\
				0\ x \notin \rationals
			\end{cases}\]

			where $\rationals$ is the set of rational numbers. Prove that $f$ is NOT Riemann integrable on $[0, 1]$. Hint: using the fact that any interval of real numbers contains both rationals and irrationals. 
		\end{em}

		To show that the function $f$ is not Riemann integrable, we must show that \[\underset{P}{sup}\{L_P(f)\} \neq \underset{P}{inf} \{U_P(f)\}\]

		Since any interval of real numbers contains both rational and irrational numbers, we know that for any partition $P$ with $N$ intervals, $m_i(f) = 0$ and $M_i(f) = 1$ for all $1 \leq i \leq N$. 

		By def'n, for any $P$ with $N$ intervals, and since $M_i(f) = 1, m_i(f) = 0$ are constants, we know that  $L_P(f) = \sum_{i = 1}^N m_i(x_i - x_{i - 1}) = m_i \sum_{i = 1}^N x_i - x_{i - 1} = m_i \cdot (x_n - x_0) = m_i = 0$ and $U_P(f) = \sum_{i = 1}^N M_i(x_i - x_{i - 1}) = M_i \sum_{i = 1}^N x_i - x_{i - 1} = M_i \cdot (x_n - x_0) = M_i = 1$. 

		Thus $\underset{P}{sup}\{L_P(f)\} = 0$ and $\underset{P}{inf} \{U_P(f)\} = 1$, so $\underset{P}{sup}\{L_P(f)\} \neq \underset{P}{inf} \{U_P(f)\}$ and the function is not Riemann integrable.

	\section{}
		\begin{em}
			Let $f$ and $g$ be both Riemann integrable on $[a, b]$. Prove that $f\ g$ is Riemann integrable on $[a, b]$. Hint: One strategy is to first show that $f^2$ is integrable if f is integrable and then use the fact that $f\ g = ((f + g)^2 - f^2 - g^2)/2$. 
		\end{em}

		\underline{Lemma: if $f$ is integrable then $f^2$ is integrable.} Since $f$ is integrable, we know that $\forall\ \varepsilon > 0\ .\ \exists\ P\ .\ U_P(f) - L_P(f) \leq \varepsilon$. We would like to show that $\forall\ \varepsilon > 0\ .\ \exists\ P\ .\ U_P(f^2) - L_P(f^2) \leq \varepsilon$. 

		Consider 

		\begin{align*}
			|f^2(x) - f^2(y)| = |(f(x) + f(y))(f(x) - f(y))| &\leq |f(x) + f(y)||f(x) - f(y)| \\
			&\leq 2B |f(x) - f(y)| \\
			&\leq 2B |M_i(f) - m_i(f)|
		\end{align*}

		But also

		\[|f^2(x) - f^2(y)| \leq |M_i(f^2) - m_i(f^2)|\]

		by the least upper bound property of $f^2$. Because $|M_i(f^2) - m_i(f^2)|$ is the least upper bound, we can see that $|M_i(f^2) - m_i(f^2)| \leq 2B|M_i(f) - m_i(f)|$. 

		Now we can consider

		\begin{align*}
			U_P(f)^2 - L_P(f^2) &= \sum_{i = 1}^N(M_i(f^2) - m_i(f^2))(x_i - x_{i - 1}) \\
			&\leq \sum_{i = 1}^N(2B(M_i(f) - m_i(f))(x_i - x_{i - 1})) \\
			&\leq 2B (U_P(f) - L_P(f)) \\
			&\leq 2B \varepsilon
		\end{align*}

		Which, by the Riemann integrability test, implies that $f^2$ is integrable if $f$ is integrable. 

		Because $f\ g = ((f + g)^2 - f^2 - g^2)/2$ and integrability is closed under addition, we know that $f\ g$ is integrable as desired.

	\section{}
		\begin{em}
			Let $f$ be a continuous function on $[a, b]$ and suppose that $\delta > 0$. 
		\end{em}

		\subsection*{a)}
			\begin{em}
				Prove that $\int_a^b f(x)dx = \lim_{\delta \to 0} \int_{a + \delta}^b f(x)dx$.
			\end{em}

			Consider a partition $P$ with $N$ intervals of size $\delta$ so $x_1 = x_0 + \delta = a + \delta$ and $b = x_N = a + N \delta$. Then $b = a + N \delta \to \delta = \frac{b - a}{N}$ so $\lim_{\delta \to 0}\int_{a + \delta}^b f(x)dx = \lim_{N \to \infty}\sum_{i = 1}^N f(x_i^*)(x_i - x_{i - 1})$ by the def'n of a Riemann sum. But $\lim_{N \to \infty}\sum_{i = 1}^N f(x_i^*)(x_i - x_{i - 1}) = \int_a^bf(x)dx$ which gives the desired result. Thus $\int_a^b f(x)dx = \lim_{\delta \to 0} \int_{a + \delta}^b f(x)dx$. This result was also proved in theorem 14.2. 

			% Literally pulled the last equality out of thin air

		\subsection*{b)}
			\begin{em}
				Prove that $\int_a^b f(x)dx = \lim_{\delta \to 0} \int_{a + \delta}^{b - \delta} f(x)dx$.
			\end{em}

			We can consider the open interval $(a, b)$ to apply theorem 14.3 which gives us the points $a_1 = a, a_K = b$ and gives us the following equality
			\[\int_a^b f(x)dx = \sum_{j = 1}^{K + 1}\lim_{\delta \to 0}\int_{a_{j - 1} + \delta}^{a_j - \delta}f(x)dx = \lim_{\delta \to 0} \int_{a + \delta}^{b - \delta} f(x)dx\]

			as desired.

\end{document}
