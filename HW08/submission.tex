\documentclass[]{article}

\author{Rushi Shah}
\date{\today}
\title{Assignment 8}

\setlength{\columnsep}{.75in}

\usepackage{amsmath}
\usepackage{amsthm}
\usepackage{amssymb}
\usepackage{mathtools}

\usepackage[margin=1in]{geometry}

\usepackage{titling}
\setlength{\droptitle}{-85pt}

% Prefix section headers with "Problem"
\renewcommand*{\thesection}{Problem~\arabic{section}}


% Note: `\to` is a synonym for `\rightarrow`
\newcommand{\integers}{\mathbb{Z}}
\newcommand{\naturals}{\mathbb{N}}
\newcommand{\reals}{\mathbb{R}}
\newcommand{\complexes}{\mathbb{C}}
\newcommand{\rationals}{\mathbb{Q}}
\newcommand{\inv}{^{-1}}
\DeclarePairedDelimiter\floor{\lfloor}{\rfloor}


\begin{document}
	\maketitle

	\section{}
		\textit{Let $f$ be a real-valued function defined as $f(x) = x^2$ on $Dom(f) = \reals$, and let $\varepsilon > 0$ be given. Find a $\delta$ so that $|x - 1| \leq \delta$ implies $|f(x) - 1| \leq \varepsilon$.}

		We want to show

		\begin{align*}
			|f(x) - 1| &\leq \varepsilon \\
			|x^2 - 1| &\leq \varepsilon \\
			|(x + 1)(x - 1)| &\leq \varepsilon \\ 
			|(x - 1 + 2))(x - 1)| &\leq \varepsilon \\
			|x - 1 + 2||x - 1| &\leq \varepsilon \\
			(|x - 1| + |2|)|x - 1| &\leq \varepsilon \\
			(\delta + 2)\delta &\leq \varepsilon \\
			\delta^2 + 2 \delta &\leq \varepsilon
		\end{align*}

		We can solve $\delta^2 + 2 \delta \leq \varepsilon$ to get an equation for $\delta$ in terms of $\varepsilon$. Take $\delta = \sqrt{\varepsilon + 1} - 1$.

	\section{}
		\textit{Prove that the real-valued function $f(x) = 1/x$ defined on $Dom(f) = [1, \infty)$ is uniformly continuous under the usual Euclidean metric $\rho(x, y) = |x - y|$ on $\reals$.}

		To show uniform continuity we would like to show that given any $\varepsilon > 0$, we can select a $\delta$ independent of $x$ such that $\rho(x, y) \leq \delta \to \rho(f(x), f(y)) \leq \varepsilon$. So consider some $\varepsilon > 0$, and $x, y$ such that $|x - y| \leq \delta$. 

		\begin{align*}
			\rho(f(x), f(y)) &\leq \varepsilon \\
			|f(x) - f(y)| &\leq \varepsilon	\\
			|1/x - 1/y| &\leq \varepsilon \\
			|\frac{y - x}{xy}| &\leq \varepsilon \\
			\frac{|y - x|}{|xy|} &\leq \varepsilon \\
			|y - x| &\leq \varepsilon \cdot |x| \cdot |y| \\
			\delta &\leq \varepsilon \cdot x \cdot y \\
			\delta &\leq \varepsilon \leq \varepsilon \cdot x \cdot y
		\end{align*}

		Where we know that $\varepsilon \leq \varepsilon \cdot x \cdot y$ since $x, y \in [1, \infty)$. Thus we can select $\delta \leq \varepsilon$ independent of $x$ to satisfy uniform continuity. For example, we can always select $\delta = \varepsilon / 2$.

	\section{}
		\textit{Let $(X, \sigma)$ be a metric space and $x_0$ a point in X. Prove that the function $f : X \to \reals$ defined as $f(x) = \sigma(x, x_0)$ is continuous under the usual Euclidean metric $\rho(x, y) = |x - y|$ on $\reals$.}

		To show continuity we would like to show that given any $\varepsilon > 0$, we can select a $\delta$ such that $\forall\ a, b\ .\ \sigma(a, b) \leq \delta \to |f(a) - f(b)| \leq \varepsilon$

		We can use the triangle inequality of $\sigma$ to note that 

		\begin{align*}
			|f(a) - f(b)| \leq \varepsilon \\
			|\sigma(a, x_0) - \sigma(b, x_0)| \leq \varepsilon \\
			|\sigma(a, b)| \leq |\sigma(a, x_0) - \sigma(b, x_0)| \leq \varepsilon \\
			\delta \leq \varepsilon
		\end{align*}

		Thus by taking $\delta \leq \varepsilon$, for example $\delta = \varepsilon / 2$, we can satisfy the criteria for continuity. 

	\section{}
		\textit{Let $(X, \rho_X)$ and $(Y, \rho_Y)$ be two metric spaces, and $f : X \to Y$ a function with domain $Dom(f) = X$ and range $Ran(f) = Y$. Prove that $f$ is continuous if and only if $f^{-1}[\tilde{Y}]$ (the preimage of $\tilde{Y}$) is open for every open set $\tilde{Y}$ in Y.}

		% Try rewriting definition of continuity in terms of open balls. Because p(y, y_0) < epsilon means that that is an open ball with radius epsilon around y_0. So you can rewrite what continuous means in terms of open balls around things existing


% Suppose $f$ is continuous. Let $\tilde{Y}$ be any open set in $Y$. It follows that for every point $y \in \tilde{Y}$ there exists an $x \in X$ such that $f(x)=y$ and that there exists an $\varepsilon>0$ s.t. $B_\varepsilon(y)\subseteq \tilde{Y}$. By the definition of a continuous function, there exists a $\delta>0$ such that all points in the pre-image of $B_\varepsilon(y)$ must be no more than a distance of $\delta$ away from eachother. That is, for every $f(x)\in B_\varepsilon(y)$ there exists a $B_\delta(x)$ that is contained entirely within $f^{-1}[\tilde{Y}]$. Therefore, $f^{-1}[\tilde{Y}]$ is an open set.

		

		\underline{If $f$ is continuous, then for every open set $\tilde{Y} \in Y$ we know $f^{-1}[\tilde{Y}]$ is open.}

		Since f is continuous, we know that for any $\varepsilon > 0$, if $f(y) \in B_\varepsilon(f(x))$ then there exists some $\delta$ such that $y \in B_\delta(x)$. Since $\tilde{Y}$ is an open set in $Y$, we know that for any $f(x) \in \tilde{Y}$, there is some $\varepsilon > 0$ such that $B_\varepsilon(f(x)) \subseteq \tilde{Y}$. Thus for any $f(x) \in \tilde{Y}$, we can take the corresponding $\varepsilon$ and plug it into the def'n of continuous to see that there exists some $\delta$ such that $B_\delta(x) \subseteq X$. This implies that every $x \in X$ has an open ball of some positive radius  that is a subset of X, which means that $f^{-1}[\tilde{Y}]$ is open.

		% f is continuous means that if you have some $x, y \in X$ and you know $y \in B_\delta(x)$ then you know that $f(y) \in B_\varepsilon(f(x))$. 

		% Recall that if $\exists\ $ a limit point $a$ not in set $A$, then $A$ is open. Also recall that a limit point $a$ is a point such that $\exists\ $ a non-constant sequence of $\{a_n\} \subseteq A$ such that $\{a_n\} \to a$. Thus we must show that there exists $\{x_n\} \subseteq f^{-1}[\tilde{Y}]$ such that $\{x_n\} \to x$ where $x \notin f^{-1}[\tilde{Y}]$. Consider an open set $\tilde{Y} \in Y$. Since this set is open, there exists some non-constant sequence $\{y_n\} \to y$ where $y \notin \tilde{Y}$ but $\{y_n\} \subseteq \tilde{Y}$. Thus, we can take the preimage of all $\{y_n\}$ to be a non-constant sequence $\{x_n\} \subseteq f^{-1}[\tilde{Y}]$. But, it is also clear that $x = f^{-1}(y) \notin f^{-1}[\tilde{Y}]$ because $y \notin \tilde{Y}$. Because $f$ is continuous, we know that$\{x_n\} \to x$ where $x \notin f^{-1}[\tilde{Y}]$. Thus, we know that $f^{-1}[\tilde{Y}]$ is open. 

		\underline{If for every open set $\tilde{Y} \in Y$ we know that $f^{-1}[\tilde{Y}]$ is open, then $f$ must be continuous.}

		We would like to show that for any $\varepsilon > 0$, there exists some $\delta$ such that $\rho_X(x, y) < \delta$ implies $\rho_Y(f(x), f(y)) < \varepsilon$. Since $\tilde{Y}$ is open, we know that for any $f(x) \in \tilde{Y}$ there exists some $\varepsilon > 0$ such that $B_\varepsilon(f(x)) \subseteq(\tilde{Y})$. Since $f^{-1}[\tilde{Y}]$ is also open, we know that for any $x$ there exists some $\delta > 0$ such that $B_\delta(x) \subseteq X$. So for any $\varepsilon > 0$, we can select $\tilde{Y} = B_\varepsilon(f(x))$, which gives us the corresponding $f^{-1}[\tilde{Y}]$ with the ball $B_\delta(x)$. The connection between this $x, \delta,$ and $\varepsilon$ satisfies the criteria for $f$ to be continuous. 

% Suppose that $f^{-1}[\tilde{Y}]$ is open for any open set $\tilde{Y}\subseteq Y$. It follows that for all $x\in f^{-1}[\tilde{Y}]$ there exists $\delta >0$ s.t. $B_\delta(x)\subseteq f^{-1}[\tilde{Y}]$. Since $\tilde{Y}$ is an open set, we also know that there exists an $\varepsilon>0$ s.t. $B_\varepsilon(f(x))\subseteq \tilde{Y}$. 
		
		% For any $f(x),f(y)\in \tilde{Y}$ such that $\rho_Y(f(x),f(y))\leq\varepsilon$, it is clear that $f(x), f(y)\in B_\varepsilon(f(x))$. Since an open ball is an open set and the pre-image of an open set is open, it follows that there exists a $\delta >0$ such that $x,y \in B_\delta(x)$ and $B_\delta(x)\subseteq f^{-1}[\tilde{Y}]$. It follows that $\rho_X(x,y)\leq \delta$. 
		
		% Therefore, for any $\varepsilon>0$, there exists a $\delta$ such that $\rho_X(x,y)\leq \delta \implies \rho_Y(f(x),f(y))\leq\varepsilon$. Therefore f is continuous.



		% As shown in class, an equivalent condition for continuity is $\forall\ \{x_n\} \subseteq X, x_n \to x_0$ implies $f(x_n) \to f(x_0)$. Consider some arbitrary $\{x_n\} \subseteq X$.

		% Thus consider an arbitrary open set $\tilde{Y} \in Y$

		% , which gives us $f : f^{-1}[\tilde{Y}] \to \tilde{Y}$. 

	\section{}
		\begin{em}
			Let $f$ be the function on $[0, 1]$ given by 

			\[f(x) = \begin{cases}
				1\ if\ x \neq \frac{1}{2} \\
				2\ if\ x = \frac{1}{2}
			\end{cases}\]

			Prove that $f$ is Riemann integrable and compute $\int_0^1 f(x) dx$. Hint: for each $\varepsilon > 0$, find a partition P so that $U_P(f) - L_P(f) \leq \varepsilon$. 
		\end{em}

		Given any $\varepsilon$, we can select the partition $P = \left \{0, \frac{1}{2} - \frac{\varepsilon}{2}, \frac{1}{2} + \frac{\varepsilon}{2}, 1 \right \}$. Thus

		\begin{align*}
			U_P(f) &= 1 \left (\frac{1}{2} - \frac{\varepsilon}{2} - 0 \right ) + 2\left (\frac{1}{2} + \frac{\varepsilon}{2} - \left (\frac{1}{2} - \frac{\varepsilon}{2}\right ) \right ) + 1\left (1 - \left (\frac{1}{2} + \frac{\varepsilon}{2} \right) \right ) \\
			&= 1 + \varepsilon
		\end{align*}

		\begin{align*}
			L_P(f) &= 1 \left (\frac{1}{2} - \frac{\varepsilon}{2} - 0 \right ) + 1\left (\frac{1}{2} + \frac{\varepsilon}{2} - \left (\frac{1}{2} - \frac{\varepsilon}{2}\right ) \right ) + 1\left (1 - \left (\frac{1}{2} + \frac{\varepsilon}{2} \right) \right ) \\
			&= 1 
		\end{align*}

		So, $U_P(f) - L_P(f) = 1 + \varepsilon - 1 = \varepsilon$, which by the Riemann integrability test, shows that $f$ is integrable. The value of $\int_0^1 f(x) dx = sup_P[ L_P(f) ] = 1$.
\end{document}
